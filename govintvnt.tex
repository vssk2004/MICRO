\section{Government Intervention}


\subsection{Indirect Taxes}

\subsubsection{Introduction to Indirect Taxes}
Indirect taxes are imposed on spending (spending to buy goods and services). They are paid
by producers to the government, but consumers contribute a part of the \emph{tax burden}. For this reason,
the tax is called `indirect' tax.
There are two types of indirect taxes:
\begin{enumerate}
    \item Excise Duties: These are taxes on specific goods, such as cigarettes, alcohol, and petrol.
    \item Taxes on all (or most) spending, such as Value Added Tax (VAT) or Goods and Services Tax (GST).
\end{enumerate}

Indirect taxes differ from direct taxes, which are imposed on income, wealth, or profits. Direct taxes are paid
directly by taxpayers to the government, unlike indirect taxes, which are collected by producers on behalf
of the government.

\subsubsection{Why do Governments Impose Excise Taxes?}
Governments impose excise taxes for several reasons:
\begin{itemize}
    \item Excise taxes are a source of government revenue. Especially for goods that are inelastic in demand,
    such as tobacco and alcohol, excise taxes can generate significant revenue.
    \item Excise taxes can be used to discourage the consumption of certain goods that have \emph{negative externalities},
    such as tobacco and alcohol.
    \item Excise taxes can also be used to correct market failures.
    \item Excise taxes can be used to redistribute income. Taxes on luxury goods like yachts, private jets,
    and high-end cars can lower after-tax income, reducing income inequality.
\end{itemize}

\subsubsection{Specific vs. Ad Valorem Taxes}
There are two main types of excise taxes:
\begin{itemize}
    \item Specific Taxes: These are fixed amounts of tax per unit of the good sold. For example, a tax of
    \euro 1 per pack of cigarettes.
    \item Ad Valorem Taxes: These are taxes based on a percentage of the price of the good sold. For example,
    a VAT of $20\%$ on most goods and services in the UK.
\end{itemize}

These taxes have different effects on prices and quantities sold. Specific taxes tend to increase the price
by a fixed amount, while ad valorem taxes increase the price by a percentage, which can lead to larger price increases
for higher-priced goods. Consider the following supply curves:

\begin{center}
\begin{minipage}{0.48\textwidth}
  \centering
  \includegraphics[scale=0.6]{SPECTAX}
\end{minipage}
\hfill
\begin{minipage}{0.48\textwidth}
  \centering
  \includegraphics[scale=0.6]{ADTAX}
\end{minipage}
\end{center}

Here, (a) shows the effect of a specific tax and (b) shows the effect of an ad valorem tax.
Now, we can observe the impact of these taxes on government revenue.
\begin{center}
\begin{minipage}{0.48\textwidth}
  \centering
  \includegraphics[scale=0.7]{GOVREV1}
\end{minipage}
\hfill
\begin{minipage}{0.48\textwidth}
  \centering
  \includegraphics[scale=0.7]{GOVREV2}
\end{minipage}
\end{center}

\medskip
In both cases:
\begin{itemize}
    \item equilibrium quantity produced and consumed falls from \(Q^*\) to \(Q_t\)
    \item equilibrium price paid by consumers rises from \(P^*\) to \(P_c\)
    \item price received by producers falls from \(P^*\) to \(P_p\), which is \(P_p = P_c - \text{tax}\)
    \item producers' revenue falls from \(P^* \times Q^*\) to \(P_p \times Q_t\)
    \item government revenue from the tax is \((P_c - P_p) \times Q_t\)
    \item there is an underallocation of resources to the production of the good; \(Q_t\) is less than the
    free market equilibrium quantity \(Q^*\)
\end{itemize}


\subsection{Impact of Indirect Taxes}

\subsubsection{Impact of Specific Tax on Social Welfare}
Recall that social surplus (or social welfare) is the sum of consumer surplus and producer surplus.
In a competitive market without government intervention, social surplus is maximized at the equilibrium quantity \(Q^*\),
indicating that allocative efficiency is achieved. What happens to social surplus and allocative efficiency
after the imposition of a specific tax?

\begin{center}
\begin{minipage}{0.48\textwidth}
  \centering
  \includegraphics[scale=0.5]{SURPLUS1}
\end{minipage}
\hfill
\begin{minipage}{0.48\textwidth}
    \centering
    \includegraphics[scale=0.5]{SURPLUS2}
\end{minipage}
\end{center}

As shown above, the imposition of a specific tax reduces consumer surplus and producer surplus.
A portion of consumer surplus becomes government revenue, while the rest is lost as triangle a.
A portion of producer surplus also becomes government revenue, while the rest is lost as triangle b.

The consumer and producer surplus that is transformed into government tax revenue comes back
to society in the form of government spending. Therefore, the after-tax social surplus
in Figure 4.4(b) is equal to after-tax consumer and producer surplus plus government revenue.

However, after-tax social surplus is less than pre-tax social surplus by the amount of
triangles a + b. The areas a + b represent social surplus that is completely lost, and is
called welfare loss or deadweight loss (DWL).

\dfn{Deadweight Loss (Welfare Loss )}{Represents welfare benefits that are lost to
society because resources are not allocated efficiently.}

The welfare loss in this case is the result of underallocation of resources to the production of
the good (underproduction). This is also indicated by MB \(>\) MC: too little of the good is produced and
consumed relative to the social optimum.

\subsubsection{Tax Burden, PED, and PES}
The tax burden (or incidence) refers to how the burden of paying a tax is divided between
consumers and producers. The division of the tax burden depends on the price elasticities of demand
and supply for the good being taxed.

\medskip
The total tax revenue generated by the tax is: \((P_c - P_p) \times Q_t\).\\
Tax burden of consumers \(= (P_c - P^*) \times Q_t\)\\
Tax burden of producers \(= (P^* - P_p) \times Q_t\)

\medskip
The division of the tax burden depends on the relative price elasticities of demand and supply. First,
we consider the incidence of the tax on consumers and producers with different price elasticities of demand.

\begin{center}
\begin{minipage}{0.48\textwidth}
  \centering
  \includegraphics[scale=0.6]{TXBD1}
\end{minipage}
\hfill
\begin{minipage}{0.48\textwidth}
    \centering
    \includegraphics[scale=0.6]{TXBD2}
\end{minipage}
\end{center}

When demand is relatively inelastic (\(0 < PED < 1\)), consumers bear a larger share of the tax burden.
Conversely, when demand is relatively elastic (\(PED > 1\)), producers bear a larger share of the tax burden.
Why does this happen?

When demand is inelastic, consumers are less responsive to price changes,
so they continue to buy the good even at higher prices (like food), leading to a larger tax burden on them.
When demand is elastic, consumers are more responsive to price changes, so they reduce their quantity
demanded significantly when the price rises (like luxury goods), leading to a larger tax burden on producers.

\begin{center}
\begin{minipage}{0.48\textwidth}
  \centering
  \includegraphics[scale=0.625]{TXBD3}
\end{minipage}
\hfill
\begin{minipage}{0.48\textwidth}
    \centering
    \includegraphics[scale=0.61]{TXBD4}
\end{minipage}
\end{center}

When supply is relatively inelastic (\(0 < PES < 1\)), producers bear a larger share of the tax burden.
Conversely, when supply is relatively elastic (\(PES > 1\)), consumers bear a larger share of the tax burden.


\subsection{Subsidies}


\subsection{Impact of Subsidies}


\subsection{Price Controls}