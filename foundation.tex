\section{Foundations of Economics}
{
\subsection{Fundamental Problem of Economics}
{
Human beings have many wants and needs. The physical objects they want or need are called 
\emph{goods} (e.g., food, clothing, books), while the non-physical activities are called 
\emph{services} (e.g., education, health care, entertainment).

The study of economics arises because people's needs and wants are unlimited, but the
\emph{resources} needed to satisfy them are limited. Resources are inputs used to produce
goods and services, and for this reason are also known as \emph{factors of production}. Factors 
of production do not exist in abundance; they are \emph{scarce}.
\dfn{Scarcity}{Scarcity is the situation in which available resources, or factors of production, are finite,
whereas wants are infinite. There are not enough resources to produce everything that human beings
need and want.}

As a result of scarcity, choices need to be made. Resource scarcity
forces society to make a choice between available alternatives.
Another important consequence of scarcity is avoiding waste in using
resources. If resources are not used effectively and are wasted,
they will end up producing less. Finally, scarcity gives rise to
\emph{opportunity cost}.
\dfn{Opportunity Cost}{Opportunity cost is defined as the value of the next
best alternative that must be given up or sacrificed in order to obtain something else.}
When a consumer chooses to use her \(\$\)100 to buy
a pair of shoes, she is also choosing not to use this
money to buy books. The foregone books are the opportunity cost.
}

\subsection{Assumptions in Model-Building}
{
Economists primarily make two assumptions when building models:
\begin{enumerate}
    \item Ceteris Paribus
    \item Rational Agents
\end{enumerate}

\dfn{Ceteris Paribus}{A Latin expression that means `other things equal'.
In the context of economics, it is saying that all other things are assumed to be constant or unchanging
in order to study the effect of one independent variable on a dependent variable.}

\dfn{Rational Agents}{Rational economic decision-making. This 
means that individuals are assumed to act in their best self-interest, trying to maximise
(make as large as possible) the satisfaction they expect to receive from their decisions.}
}

\subsection{What is Microeconomics?}
{
\dfn{Microeconomics}{Microeconomics is concerned with the behaviour of
consumers, firms and resource owners, who are the most
important economic decision-makers in a market economy.}
}
}


\newpage


\section{Demand and Supply}
{
\subsection{What is a Market?}
{
It is easiest to understand what a market is and how it works by dividing
individual economic units into two broad groups, according to function,
\emph{buyers} and \emph{sellers}.

Buyers purchase goods and services. Usually, there are two types of buyers:
consumers and firms. Consumers purchase regular goods and service while firms
purchase labor, capital, and raw materials that they use to produce goods and services.

Sellers sell goods and services. Usually, there are three types of sellers: firms,
resource owners, and workers. Firms sell their goods and services, resource owners
rent land or sell mineral resources to firms, and workers sell their labor services.

\dfn{Market}{A market is an arrangement where buyers and sellers meet to
carry out an exchange which determines the price of a product.}
}

\subsection{Competitive and Non-Competitive Markets}
{
A \emph{perfectly competitive market} has many buyers and sellers, so that
no single buyer or seller has any impact on price. Most agricultural markets are
close to being perfectly competitive.

Some markets contain many producers but are \emph{non-competitive}; that is, individual firms can
jointly affect the price. The world oil market is one example. Since the early
1970s, that market has been dominated by the OPEC cartel.
}
}